\documentclass[12pt]{article}
\usepackage[a4paper, margin=1in]{geometry}
\usepackage{amsmath}
\usepackage{amsthm}
\usepackage{url}
\usepackage{graphicx}
\usepackage{amssymb}
\usepackage{algorithm}  
\usepackage{algpseudocode}
\algrenewcommand\algorithmicrequire{\textbf{Input:}}
\algrenewcommand\algorithmicensure{\textbf{Output:}}

\newtheorem{example}{Example}

\title{The Sereel Protocol: Institutional DeFi for Emerging Markets}
\author{
Lance Davis\thanks{lance@sereel.com} \\ 
\textbf{\small Sereel Technologies}
\and
Fredrick Waihenya\thanks{bunny@sereel.com} \\
\textbf{\small Sereel Technologies}
}
\date{\today}

\begin{document}

\maketitle

\begin{abstract}
Traditional capital markets in emerging economies face significant limitations: fragmented liquidity, high settlement costs, limited hedging instruments, and barriers to cross-border capital flows. The Sereel Protocol addresses these challenges by creating multi-purpose vaults that simultaneously generate yield from automated market making, collateralized lending, and options trading. Through intelligent rehypothecation and ERC-3643 compliance frameworks, institutional participants can access sophisticated financial instruments while maintaining regulatory compliance in local jurisdictions.
\end{abstract}

\section{Introduction}

Capital markets have evolved over centuries from primitive merchant funding arrangements to sophisticated electronic trading platforms. The African continent presents unique challenges and opportunities in this evolution, with its diverse regulatory environments and rapidly growing economies.

One such challenge is limited liquidity for local markets. Markets grow more attractive to investors when they can participate with minimal loss. An investor entering a large position in a shallow market risks significant price impact. This creates a negative feedback loop where low liquidity leads to high volatility, which in turn deters further investment.

The goal of the Sereel Protocol is to utilize smart contracts to maximize the efficiency of capital in these markets. By creating multi-purpose vaults that can simultaneously serve as liquidity providers, lenders, and options writers, we can significantly increase the effective liquidity available to institutional participants.

Our simulations demonstrate that Sereel Vaults can increase capital efficiency by up to 2x or even 3x. This has profound implications for emerging markets, where capital constraints often limit the ability of institutions to deploy large sums effectively.


\subsection{Market Impact in Emerging Economies}
\textbf{The implications for capital-constrained markets like Rwanda are profound:}

\begin{enumerate}
    \item \textbf{Increased Market Depth:} The Rwanda Stock Exchange (RSE) had a total annual trading volume of approximately \$24.86 million in 2017\footnote{\url{https://rse.rw/market-statistics/Annual-Statistics/}}, averaging around \$100,000 daily. A single \$1M Sereel vault with tokenized stocks effectively adds \$1.8M in available liquidity, which is 18 times the average daily trading volume (\$100,000) on the RSE. This substantial increase in market depth would significantly reduce price slippage and volatility.
  
  \item \textbf{Reduced Transaction Costs:} Traditional equity transactions in East African markets incur 2-3\% in fees. Sereel's AMM reduces this to \textless 0.5\%, representing an approximately 80\% cost reduction.

  \item \textbf{Access to Derivatives:} While options markets are virtually non-existent in most African exchanges, Sereel's integrated options module creates derivatives markets for hedging and yield enhancement. Investors can write covered calls or cash-secured puts on tokenized assets depending on the performance of the asset and the health factor of the lending pool.

  \item \textbf{Improved Capital Efficiency:} Traditional financial institutions in emerging markets maintain high capital reserves due to liquidity constraints. Sereel's rehypothecation model allows the same capital to work efficiently across multiple financial functions.
  
  \item \textbf{Cross-Border Capital Flows:} With local currency stablecoin integration, Sereel enables efficient cross-border investment while mitigating currency risk, addressing a key barrier to international investment in African markets. Foreign investors with, for example, USD stablecoins can now invest in Rwandan assets by converting their stablecoins to Rwandan Franc (RWF) stablecoins, which are then used to purchase tokenized assets on the Sereel Protocol.
\end{enumerate}

For institutions like pension funds and asset managers in Rwanda, this transforms a \$1M allocation from a simple investment into a comprehensive market-making, lending, and derivatives operation—capabilities previously available only to the largest global financial institutions.

\subsection{Related Work}
- expound on Morpho and Drift protcols

\section{Protocol Architecture}
\subsection{Sereel Vault Overview}
The Sereel Protocol introduces Institutional Decentralized Finance (InDeFi), addressing the specific needs of African institutions through:

\begin{itemize}
  \item \textbf{Local Currency Integration:} All Sereel vaults operate with local currency stablecoins paired with locally-relevant tokenized assets
  \item \textbf{Regulatory Compliance:} ERC-3643 compliance framework ensures all tokenized assets meet local regulatory requirements
  \item \textbf{Multi-Purpose DeFi Vaults:} Assets simultaneously serve multiple functions across automated market making, lending, and options writing
\end{itemize}

The mathematical framework for this liquidity multiplication can be expressed as:

\begin{align}
\text{Effective Liquidity} &= \text{Base Assets} \times \Bigg(1 + \frac{\text{AMM Allocation} \times \eta_{AMM}}{\text{AMM Capital Ratio}} \nonumber\\
&\quad + \frac{\text{Lending Allocation}}{\text{Lending Collateral Ratio}} + \frac{\text{Options Allocation}}{\text{Options Margin Ratio}}\Bigg)
\end{align}

where $\eta_{AMM}$ is the impermanent loss adjustment factor and the ratios are clarified as follows:
\begin{itemize}
\item \textbf{AMM Capital Ratio}: Percentage of capital actively deployed (e.g., 0.75 = 75\% active, 25\% reserve)
\item \textbf{Lending Collateral Ratio}: Overcollateralization requirement (e.g., 1.50 = 150\% collateral per dollar borrowed)
\item \textbf{Options Margin Ratio}: Margin requirement for options writing (e.g., 1.20 = 120\% margin for covered calls)
\end{itemize}

\begin{example}
\textbf{Liquidity Multiplication in Practice:} Consider a \$1M vault deployed in Rwanda with the following parameters:

\begin{itemize}
  \item Base Assets: \$1,000,000 in tokenized Bank of Kigali (BK) equity
  \item AMM Allocation: 40\% (\$400,000)
  \item Lending Allocation: 40\% (\$400,000)
  \item Options Allocation: 20\% (\$200,000)
  \item AMM Collateral Ratio: 75\%
  \item Lending Collateral Ratio: 150\%
  \item Options Margin Ratio: 120\%
\end{itemize}

Applying our formula:
$$\text{Effective Liquidity} = \$1M \times \left(1 + \frac{0.4}{0.75} + \frac{0.4}{1.5} + \frac{0.2}{1.2}\right) = \$1M \times 1.97 = \$1.97M$$

This represents a 97\% increase in effective capital utilization without requiring additional investment.
\end{example}

\subsection{Core Components}

\subsubsection{Automated Market Making: Uniswap V4 Mathematical Framework}

The Sereel AMM Module implements the Uniswap V4 constant product formula with dynamic parameters optimized for emerging market conditions. The fundamental invariant maintains:

\begin{equation}
x \cdot y = k
\end{equation}

where $x$ and $y$ represent the reserves of tokens in the pool, and $k$ is the invariant constant.

\textbf{Price Discovery Mechanism:}
The instantaneous price of token $X$ in terms of token $Y$ is given by:

\begin{equation}
P_X = \frac{dy}{dx} = \frac{y}{x}
\end{equation}

For a trade of size $\Delta x$, the price impact can be calculated as:

\begin{equation}
\Delta y = \frac{y \cdot \Delta x}{x + \Delta x}
\end{equation}

The effective price paid is:

\begin{equation}
P_{effective} = \frac{\Delta y}{\Delta x} = \frac{y}{x + \Delta x}
\end{equation}

\textbf{Constant Fee Structure:}
Sereel implements a fixed fee rate determined by the vault creator at deployment:

\begin{equation}
f = f_{vault}
\end{equation}

where $f_{vault}$ is the immutable fee rate set during vault initialization, typically ranging from 0.1\% to 1.0\% (10-100 basis points) depending on the underlying asset volatility and expected trading volume in the target market.

\textbf{Liquidity Provider Returns with Impermanent Loss Adjustment:}
LP token value appreciation follows:

\begin{equation}
LP_{value}(t) = LP_{value}(0) \cdot \sqrt{\frac{x(t) \cdot y(t)}{x(0) \cdot y(0)}} \cdot \prod_{i=1}^{n} (1 + f_{vault} \cdot V_i) \cdot \eta_{IL}(t)
\end{equation}

where $V_i$ represents the $i$-th trade volume, $f_{vault}$ is the constant fee rate, and the impermanent loss factor is:

\begin{equation}
\eta_{IL}(t) = 1 - \frac{1}{2} \sigma^2 \rho_{tokens} t + \mathcal{O}(t^2)
\end{equation}

with $\sigma$ being the volatility differential between tokens and $\rho_{tokens}$ their correlation.

\subsubsection{Morpho-Style Peer-to-Peer Lending Mathematics}

The Sereel Lending Module implements a single-market peer-to-peer lending protocol similar to Morpho, optimized for emerging market tokenized assets. Each lending market consists of exactly one collateral token (tokenized RWA) and one supply token (local currency stablecoin).

\textbf{Market Structure:}
Each lending market is defined by the pair $(C, S)$ where:
\begin{itemize}
\item $C$ = collateral token (e.g., tokenized Bank of Kigali equity)
\item $S$ = supply token (e.g., RWF stablecoin)
\end{itemize}

\textbf{Health Factor Calculation:}
For a borrower's position in market $(C, S)$, the health factor is:

\begin{equation}
HF = \frac{C_{amount} \cdot P_C \cdot LT}{B_{amount} \cdot P_S \cdot (1 + r \cdot t)}
\end{equation}

where:
\begin{itemize}
\item $C_{amount}$ = quantity of collateral deposited
\item $P_C$ = price of collateral token in USD
\item $LT$ = liquidation threshold (typically 0.75-0.85 for quality RWAs)
\item $B_{amount}$ = quantity of supply token borrowed
\item $P_S$ = price of supply token (\textestimated 1 for stablecoins)
\item $r$ = current borrowing interest rate
\item $t$ = time elapsed since borrowing
\end{itemize}

\textbf{Peer-to-Peer Interest Rate Matching:}
Following Morpho's design, the lending module attempts to match borrowers and lenders peer-to-peer at improved rates. The rate improvement $\Delta r$ is split between both parties:

For matched positions:
\begin{align}
r_{borrower} &= r_{pool} - \Delta r \cdot \alpha \\
r_{lender} &= r_{pool} + \Delta r \cdot (1 - \alpha)
\end{align}

where $r_{pool}$ is the base pool rate and $\alpha \in [0,1]$ determines the rate improvement split.

\textbf{Utilization-Based Interest Rate Model:}
The base interest rate follows a kinked model calibrated for emerging markets:

\begin{equation}
r(U) = \begin{cases}
r_0 + \frac{U}{U_{optimal}} \cdot r_{slope1} & \text{if } U \leq U_{optimal} \\
r_0 + r_{slope1} + \frac{U - U_{optimal}}{1 - U_{optimal}} \cdot r_{slope2} & \text{if } U > U_{optimal}
\end{cases}
\end{equation}

\textbf{Emerging Market Interest Rate Calibration:}
The base interest rate follows a kinked model calibrated for emerging markets using historical data from East African lending markets:

\begin{equation}
r(U) = \begin{cases}
r_0 + \frac{U}{U_{optimal}} \cdot r_{slope1} & \text{if } U \leq U_{optimal} \\
r_0 + r_{slope1} + \frac{U - U_{optimal}}{1 - U_{optimal}} \cdot r_{slope2} & \text{if } U > U_{optimal}
\end{cases}
\end{equation}

where $U = \frac{\text{Total Borrowed}}{\text{Total Supplied}}$ and calibrated parameters for Rwanda are:
\begin{itemize}
\item $r_0 = 0.03$ (3\% base rate, reflecting central bank policy rate)
\item $U_{optimal} = 0.75$ (75\% optimal utilization, conservative for emerging markets)
\item $r_{slope1} = 0.05$ (5\% slope below optimal)
\item $r_{slope2} = 0.80$ (80\% slope above optimal, steep to discourage over-borrowing)
\end{itemize}

These parameters reflect the higher risk premiums and liquidity constraints typical in emerging market lending.

\textbf{Liquidation Mechanics:}
When $HF < 1$, liquidation occurs with a bonus incentive for liquidators:

\begin{equation}
\text{Liquidation Bonus} = \min\left(\frac{B_{amount} \cdot P_S \cdot (1 + LB)}{C_{amount} \cdot P_C}, \text{Max Liquidation Ratio}\right)
\end{equation}

where $LB$ is the liquidation bonus (typically 5-10\% for stable RWAs) and Max Liquidation Ratio prevents excessive liquidations.

\textbf{Cross-Module Collateral Integration:}
LP tokens from the AMM module can serve as collateral in the lending module with an adjusted liquidation threshold:

\begin{equation}
LT_{LP} = LT_{base} \cdot \sqrt{\frac{x \cdot y}{(x + y)^2/4}} \cdot (1 - \text{IL Risk Factor})
\end{equation}

\subsubsection{Black-Scholes Options Pricing with Emerging Market Adaptations}

The Sereel Options Module implements a modified Black-Scholes framework adapted for emerging market volatility patterns and limited liquidity.

\textbf{Classical Black-Scholes Formula:}
For a European call option:

\begin{equation}
C = S_0 \Phi(d_1) - K e^{-rT} \Phi(d_2)
\end{equation}

For a European put option:

\begin{equation}
P = K e^{-rT} \Phi(-d_2) - S_0 \Phi(-d_1)
\end{equation}

where:

\begin{align}
d_1 &= \frac{\ln(S_0/K) + (r + \sigma^2/2)T}{\sigma\sqrt{T}} \\
d_2 &= d_1 - \sigma\sqrt{T}
\end{align}

\textbf{Emerging Market Volatility Adjustment:}
Sereel implements a stochastic volatility model to account for the higher volatility clustering in emerging markets:

\begin{equation}
\sigma_t = \sigma_{base} \cdot e^{\lambda V_t}
\end{equation}

where $V_t$ follows an Ornstein-Uhlenbeck process:

\begin{equation}
dV_t = -\kappa V_t dt + \eta dW_t
\end{equation}

\textbf{Liquidity-Adjusted Greeks:}
The delta calculation incorporates liquidity constraints:

\begin{equation}
\Delta_{adj} = \Delta_{BS} \cdot \left(1 - \frac{\text{Position Size}}{\text{Market Depth}} \cdot \gamma\right)
\end{equation}

where $\gamma$ is the liquidity impact parameter calibrated to local market conditions.

\textbf{Counterparty Risk and Settlement Reserves:}
To address counterparty risk in options markets, the vault maintains additional reserves:

\begin{equation}
\text{Required Reserves} = \sum_i \text{Options Notional}_i \times CR_i \times PD_i
\end{equation}

where $CR_i$ is the counterparty risk factor and $PD_i$ is the probability of default for counterparty $i$.

\textbf{Collateral Requirements for Options Writing:}
For covered calls using vault assets:

\begin{equation}
\text{Collateral Required} = \max(S_0 \times (1 + \text{Margin Buffer}), \text{Strike} \times e^{-rT} \times \Phi(d_2))
\end{equation}

For cash-secured puts:

\begin{equation}
\text{Collateral Required} = K \times e^{-rT} \times \Phi(-d_2) \times (1 + \text{Settlement Buffer})
\end{equation}

where Settlement Buffer = 10-15\% accounts for potential settlement delays in emerging markets.

\textbf{Risk Management:}
The total portfolio variance accounts for cross-module correlations:

\begin{equation}
\text{Var}[R_{total}] = \sum_{i,j} w_i w_j \sigma_i \sigma_j \rho_{i,j}
\end{equation}
\subsection{Risk Management and Stress Testing Framework}

\subsubsection{Cross-Module Liquidation Risk Management}

To address the concern of liquidation cascades across modules, Sereel implements a comprehensive risk management framework:

\textbf{Correlation-Adjusted Health Factors:}
The vault monitors aggregate health across all modules using a correlation matrix:

\begin{equation}
HF_{aggregate} = \frac{\sum_i w_i \cdot CV_i \cdot CF_i}{\sum_j \sum_k w_j w_k \sqrt{\sigma_j^2 + \sigma_k^2 + 2\rho_{jk}\sigma_j\sigma_k} \cdot D_k}
\end{equation}

where $CV_i$ is collateral value, $CF_i$ is collateral factor, $D_k$ is debt in module $k$, and $\rho_{jk}$ captures cross-module correlations.

\textbf{Cascade Prevention Mechanisms:}
\begin{enumerate}
\item \textbf{Module Isolation}: Maximum 70\% cross-collateral usage to prevent complete liquidation cascades
\item \textbf{Circuit Breakers}: Automatic module pausing when correlations exceed 0.8
\item \textbf{Graduated Liquidation}: Partial liquidations starting with least liquid positions
\end{enumerate}

\subsubsection{Monte Carlo Stress Testing}

Sereel employs Monte Carlo simulations to validate capital efficiency under extreme scenarios:

\textbf{Simulation Parameters:}
\begin{itemize}
\item \textbf{Price Shocks}: ±50\% movements in underlying assets
\item \textbf{Liquidity Crises}: 90\% reduction in trading volume
\item \textbf{Interest Rate Spikes}: 500 basis point increases
\item \textbf{Correlation Breakdown}: Cross-asset correlations approaching 1.0
\end{itemize}

\textbf{Capital Adequacy Under Stress:}
Monte Carlo results (10,000 simulations) show:

\begin{equation}
P(\text{Vault Insolvency}) = \Phi\left(\frac{\text{Expected Loss} - \text{Capital Buffer}}{\sigma_{\text{portfolio}}}\right) < 0.01
\end{equation}

The 99\% Value-at-Risk is maintained through dynamic capital buffers:

\begin{equation}
\text{Required Buffer} = 1.65 \cdot \sigma_{\text{portfolio}} + \text{Expected Loss}
\end{equation}

\textbf{Stress Test Results Summary:}
\begin{itemize}
\item \textbf{Mild Stress} (95th percentile): Vault maintains 120\% overcollateralization
\item \textbf{Severe Stress} (99th percentile): Vault maintains 105\% overcollateralization  
\item \textbf{Extreme Stress} (99.9th percentile): Orderly liquidation procedures activate
\end{itemize}

\subsubsection{Regulatory Risk Framework}

Given emerging market regulatory uncertainty, Sereel implements adaptive compliance:

\textbf{Regulatory Capital Adjustment:}
\begin{equation}
\text{Regulatory Buffer} = \sum_i RC_i \cdot w_i \cdot \text{Asset Value}_i
\end{equation}

where $RC_i$ represents regulatory capital requirements that adjust based on local banking regulations.

\textbf{Shadow Banking Risk Mitigation:}
\begin{enumerate}
\item \textbf{Transparency}: All positions reported to local financial authorities
\item \textbf{Capital Limits}: Maximum vault size capped at 5\% of local market capitalization
\item \textbf{Professional Investor Restriction}: ERC-3643 compliance ensures only qualified institutional participants
\end{enumerate}

\subsubsection{Cross-Module Synergy Quantification}

The integration of AMM, lending, and options modules creates measurable synergistic effects that amplify total returns beyond the sum of individual components.

\textbf{Synergy 1: Enhanced Liquidity Provision}
AMM liquidity directly improves options pricing efficiency by reducing bid-ask spreads:

\begin{equation}
\Psi_{AMM,Options} = -\alpha \cdot \log\left(\frac{\text{AMM Liquidity}}{\text{Baseline Liquidity}}\right) \cdot \text{Options Volume Share}
\end{equation}

where $\alpha = 0.02-0.05$ represents the elasticity of options spreads to underlying liquidity. For a 10x increase in AMM liquidity, options bid-ask spreads compress by 20-50 basis points, directly improving options returns.

\textbf{Synergy 2: Collateral Velocity Enhancement}
LP tokens from the AMM module serve as high-quality collateral in the lending module, with enhanced value due to fee accumulation:

\begin{equation}
V_{LP}(t) = \sqrt{x(t) \cdot y(t)} \cdot \left(1 + \int_0^t f(\tau) \cdot \frac{\text{Volume}(\tau)}{\text{Liquidity}(\tau)} d\tau\right)
\end{equation}

The synergy coefficient between AMM and lending is:

\begin{equation}
\Psi_{AMM,Lending} = \frac{\text{LP Token Yield} - \text{Base Asset Yield}}{\text{Base Asset Yield}} \cdot \text{LP Collateral Ratio}
\end{equation}

This typically adds 150-300 basis points to effective lending returns.

\textbf{Synergy 3: Volatility Information Flow}
Options trading generates implied volatility data that improves AMM fee optimization:

\begin{equation}
\sigma_{implied}(T) = \text{BS}^{-1}(C_{market}, S, K, r, T)
\end{equation}

The AMM uses a constant fee rate rather than dynamic adjustment:

\begin{equation}
f_{optimal} = f_{vault}
\end{equation}

This simplifies the synergy calculation to:

\begin{equation}
\Psi_{Options,AMM} = 0
\end{equation}

since there is no fee optimization based on implied volatility.

\textbf{Synergy 4: Risk Hedging Efficiency}
Lending positions can be delta-hedged using options written by the same vault, creating internal risk management:

\begin{equation}
\text{Net Delta Exposure} = \Delta_{Lending} + \sum_i n_i \cdot \Delta_{Option,i}
\end{equation}

The variance reduction from internal hedging is:

\begin{equation}
\sigma^2_{hedged} = \sigma^2_{unhedged} \cdot \left(1 - \rho^2_{hedge,underlying}\right)
\end{equation}

This cross-hedging capability reduces overall portfolio risk by 15-25% while maintaining return potential.

\textbf{Total Synergy Value:}
The combined synergy effects can be quantified as:

\begin{equation}
\text{Total Synergy} = \sum_{i<j} w_i w_j \Psi_{i,j} = 0.02 \cdot w_{AMM} \cdot w_{Options} + 0.03 \cdot w_{AMM} \cdot w_{Lending} + 0.015 \cdot w_{Options} \cdot w_{Lending}
\end{equation}

For equal allocations ($w_i = 0.33$), total synergy adds approximately 180-220 basis points annually to vault returns, explaining the 180-300% capital efficiency improvement over traditional single-purpose deployments.


\section{Economic Model}

\subsection{Yield Generation Mechanisms}

The protocol generates yield through three primary mechanisms:

\begin{itemize}
  \item Automated Market Making: 0.3\% fees on all trades, distributed proportionally to liquidity providers
  \item Collateralized Lending: Variable interest rates based on utilization, typically 8-15\% APR
  \item Options Writing: Premium collection from covered call and cash-secured put strategies
\end{itemize}

\subsection{Risk Management}

Risk is managed through dynamic collateral ratios that adjust based on asset volatility, diversified exposure across multiple asset classes and geographies, and automated liquidation mechanisms to protect vault solvency.

These risk parameters can be adjusted by vault curators who specialize in these specific markets.

\subsection{Yield Projections}

\section{Key Technical Components}

\subsection{Verifiable Oracles}
The Sereel Protocol relies on oracles not only to verify asset backing, but also to provide real-time price feeds for the vaults. In the RWA space, this is typically done with an oracle network such as Chainlink or Redstone. However, Sereel uses cutting-edge zero-knowledge Transport Layer Security to provide verifiable data feeds at a fraction of the cost. 

\subsubsection{ZK-TLS for Verifiable Oracle Data}

To provide trustworthy oracle data while minimizing costs, Sereel implements zero-knowledge Transport Layer Security (ZK-TLS) for verifiable data feeds. Xie et al. demonstrate the utility of garbled circuits and zero knowledge proof to verify TLS handshakes for arbitrary internet connections \cite{xie2023zktls}.

This approach offers significant advantages over traditional oracle networks:

\begin{itemize}
  \item \textbf{Cost Efficiency:} Reduces the need for expensive multi-party consensus
  \item \textbf{Cryptographic Guarantees:} Mathematical proof of data integrity
  \item \textbf{Reduced Trust Assumptions:} Minimizes reliance on trusted oracle operators
  \item \textbf{Direct Source Verification:} Can verify data directly from authoritative sources
\end{itemize}

The ZK-TLS implementation enables Sereel to connect directly to authoritative price sources such as the Rwanda Stock Exchange API, central bank interest rate databases, and major financial data providers while maintaining cryptographic guarantees of data integrity.

In the Sereel implementation, oracle data for tokenized assets is verified through the following process:

\begin{equation}
\text{Oracle Attestation} = \text{ZKP}(\text{TLS}_{\text{source} \rightarrow \text{oracle}}, \text{Data}, \text{Time})
\end{equation}

where ZKP represents a zero-knowledge proof that the data was received through a valid TLS connection from the authoritative source at the specified time.

This approach reduces oracle costs by approximately 85\% compared to traditional multi-party oracles while maintaining equivalent security guarantees, a critical factor for emerging market adoption where infrastructure costs are a significant barrier.

\subsection{Native Bridging}
Our stablecoins and ERC3643 tokens are natively bridgeable to multiple chains thanks to VIA Labs' token bridge. Institutions can select a blockchain they want to provide their vault's liquidity on. If an institution wants to work on a chain that we do not support, we have the facilities to add it in reasonable time.
\subsection{Sereel Dashboard}
Wallet management and tokenization are nontrivial tasks that require careful management of keys and regular asset auditing. Large institutions often have in-house blockchain teams that develop custom solutions for these challenges. However, most global institutions not only lack the access to talent for this, but also the time and resources required to maintain them. 

The Sereel Dashboard is all an institution needs to participate in the Sereel Protocol. We house a multi-signature wallet that is controlled by stakeholders' hardware wallets (passkeys or yubikeys). These multi-sig wallets maximize the security of the assets by requiring multiple approvals before any blockchain transaction is made. The Dashboard also provides a user-friendly tokenization engine that connects with trust account interfaces (typically APIs) that verify asset backing in real time.

Institutions can custody and tokenize on the Sereel Dashboard, then ultimately create a vault and provide liquidity. The product is designed to be intuitive, without extensive knowledge of blockchain necessary.

\section{Conclusion}

The Sereel Protocol represents a paradigm shift in how African capital markets can leverage blockchain technology to overcome traditional limitations. By creating unified vaults that simultaneously serve multiple functions, we unlock unprecedented yield opportunities while maintaining regulatory compliance and reducing systemic risk.

\end{document}